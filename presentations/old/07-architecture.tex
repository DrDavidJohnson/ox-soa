\newif\iffaked \fakedfalse
\iffaked
\documentclass{sepslide-soa-faked} % jim has mucked up xdvi settings
\else
\documentclass{sepslide-soa}
\fi

\title{Software architecture}
\topic{7}{Software architecture}[JungleGreen]
\indexfile{00-index.pdf}

\begin{document}

\begin{slide}
  \Title
\end{slide}

\begin{slide}
  \Contents
\end{slide}

\begin{slide}
  \Heading{Software architecture}
  \begin{itemize}
  \item design beyond the level of algorithms and data structures
  \item as abstract datatypes themselves are design beyond the level
    of individual statements
  \item gross organization; global control structure; protocols for
    communication, synchronization and data access; physical
    distribution\ldots 
  \item Garlan and Shaw, 1994, and much subsequent work \\
    eg architectural \emph{patterns}
  \item orthogonal to underlying technology: WS, CORBA\ldots
  \end{itemize}
\end{slide}

\begin{slide}
  \Subheading{Vitruvian Triad: firmitas, utilitas, venustas}
\begin{minipage}{0.6\textwidth} \raggedright
\vbox to 0.8\textheight{
\begin{itemize}
\item Vitruvius: Roman, 1st~century BC
\item \textit{Ten Books on Architecture}
\item function (utilitas, `utility')
\item form (venustas, `beauty')
\item fabrication (firmitas, `soundness')
\item architectural principles
\item retrospective
\item perhaps a bit idealist % http://graal.ewi.utwente.nl/WhitePapers/Vitruvius/Vitruvius.htm, "Debunking Vitruvius: An Anti-Hero for ICT Architects"
\end{itemize}
\vfil}
\end{minipage}\quad\begin{minipage}{0.35\textwidth}
\includegraphics[height=0.8\textheight]{diagrams/Vitruvian.eps}
\end{minipage}
\end{slide}

\begin{slide}
\Subheading{The 4+1 View Model of Architecture}
\begin{quote}
\includegraphics[width=0.75\textwidth]{diagrams/fourplusone.eps}
\end{quote}
(Philippe Kruchten, \textit{IEEE Software} 12(6) 1995)
\end{slide}

\begin{slide}
\Subheading{Models and metamodels}
\includegraphics[height=0.8\textheight]{diagrams/meta.eps}
\end{slide}

\begin{slide}
\Subheading{UML metamodel (part)}
\begin{quote}
\includegraphics[height=0.8\textheight]{diagrams/01-09-67-fig2-6.eps}
\end{quote}
\end{slide}

\begin{slide}
\Subheading{Components and connectors}
\begin{itemize}
\item \emph{components}: sequential, single-threaded \emph{computations}
\item \emph{connectors}: concurrent, multi-threaded \emph{coordination}
\item connectors should be first-class entities (cf workflow)
\item paradigm of system assembly by \emph{superimposition}
\item communication, coordination, monitoring, control
\end{itemize}
\end{slide}

\begin{slide}
\Subsubheading{Counter-example}
\begin{quote}
\includegraphics[width=0.8\textwidth]{diagrams/counter}
\end{quote}
User presses button; counter steps; label displays counter value.
\end{slide}

\begin{slide}
\Subsubheading{Connectors for coordination}
\begin{quote}
\includegraphics[width=0.8\textwidth]{diagrams/connections}
\end{quote}
\end{slide}

\begin{slide}
\Subsubheading{Visual programming}
\begin{quote}
\includegraphics[height=0.8\textheight]{diagrams/vpe}
\end{quote}
\end{slide}

\begin{slide}
\Subheading{Product line architectures}
\begin{itemize}
\item ``set of programs studied in terms of the common and special properties of the individual members'' (Parnas)
\item family of related but different products
\item separate the generic from the specific
\item eg mobile phones
\item eg engine management
\end{itemize}
\end{slide}

\begin{slide}
\Subheading{BMW car configurator}
\begin{quote}
\includegraphics[height=0.8\textheight]{diagrams/bmw.eps}
\end{quote}
(idea from Roberto Lopez on SPL)
\end{slide}

\begin{slide}
  \Heading{Client--server}
  \begin{itemize}
  \item most common and familiar architecture for distributed systems
  \item \emph{server} offers a set of services, and waits for service
    requests
  \item \emph{client} sends request to server, awaits response
  \item client is triggering process, initiates synchronization, may
    be finite; server is reactive, typically runs indefinitely
  \item multi-tier client--server architectures (eg with client-side
    proxies and server-side gateways) are \emph{layered}\ldots
  \end{itemize}
\end{slide}

\begin{slide}
  \Heading{Layered}
\begin{center}
\includegraphics[scale=0.7]{diagrams/layered.eps}
\end{center}
\end{slide}
\begin{slide}
  \begin{itemize}
  \item hierarchical organization
  \item each layer provides a \emph{service} to the layer above, and is a
    \emph{client} of the layer below
  \item strictly, only adjacent layers interact (although a layer
    may import certain services and re-export them); each layer
    implements a \emph{virtual machine}
  \item loosely, upper layer may access any lower layer; then layers
    are translucent rather than opaque
  \item eg layered communications protocols: OSI seven-layer model
  \item challenges: often efficiency considerations encourage breaking
    abstraction boundaries
  \end{itemize}
\end{slide}

\begin{slide}
  \Heading{Pipes and filters}
\begin{center}
\includegraphics[width=\textwidth]{diagrams/pipes+filters.eps}
\end{center}
  \begin{itemize}
  \item \emph{filter} reads a stream of data from input(s), writes a stream
    of data to output(s)
  \item \emph{pipe} connects output of one filter to input of another
  \end{itemize}
\end{slide}
\begin{slide}
  \begin{itemize}
  \item filter typically applies a local transformation to the data
  \item execution usually incremental, on demand: filter may start
    writing before finishing reading, so filters may compute in
    parallel
  \item filters are independent: no shared state, no dependence on
    characteristics of upstream or downstream neighbours
  \item specializations: linear \emph{pipelines}, \emph{bounded}
    buffering, \emph{typed} data, degenerate \emph{batched sequential}
  \item eg Unix processes; compiler phases; signal processing;
    systolic arrays
  \end{itemize}
\end{slide}

\begin{slide}
  \Heading{Event-based, implicit invocation}
\begin{center}
\includegraphics[height=0.8\textheight]{diagrams/event-based.eps}
\end{center}
\end{slide}
\begin{slide}
  \begin{itemize}
  \item instead of explicitly invoking a (perhaps remote) procedure,
    component triggers an \emph{event}
  \item other components register interest in such events, perhaps by
    specifying a \emph{callback}
  \item when event is triggered, \emph{event manager} (or perhaps the
    triggering component itself) invokes all appropriate registered
    callbacks
  \item event source needs no knowledge of event sinks
  \item challenges: non-determinacy, asynchrony, passing data
  \item eg database triggers, user interfaces, the \textsc{Observer}
    pattern, `pub--sub', 
    EDA, % event-driven architecture
    CEP % complex event processing
  \end{itemize}
\end{slide}

\begin{slide}
  \Subheading{Message-oriented}
  \begin{itemize}
  \item decouples components by interposing \emph{message queues}
  \item induces \emph{asynchronous communication}: components need not
    be simultaneously connected
  \item message consumers often notified by events
  \item may require (and typically provides) persistent storage for
    queued messages, improving reliability
  \item eg MQSeries, MSMQ, JMS, WS-ReliableMessageing
  \end{itemize}
\begin{center}
\includegraphics[width=\textwidth]{diagrams/message-queue.eps}
\end{center}
\end{slide}

\begin{slide}
  \Subheading{Complex event processing}
  \begin{itemize}
  \item inference of higher-level events from lower-level
  \item eg (bells + tux + gown + rice) nearby \maths{\Rightarrow} wedding
  \item abstraction, correlation, causality, timing
  \item algorithmic trading, fraud detection, embedded systems
  \item see \url{www.complexevents.com}
  \end{itemize}
\end{slide}

\begin{slide}
  \Heading{Repository}
\begin{center}
\includegraphics[width=\textwidth]{diagrams/blackboard.eps}
\end{center}
\end{slide}
\begin{slide}
  \begin{itemize}
  \item central \emph{store} maintains global state
  \item independent \emph{knowledge sources} interact with central store
  \item KS responds opportunistically when stored state makes it applicable
  \item (alternatively, actions may be triggered by input types; then
    more like a database with stored procedures)
  \item eg Blackboard systems (HEARSAY-II speech recognizer)
  \end{itemize}
\end{slide}

\begin{slide}
  \Subheading{Tuple spaces}
  \begin{itemize}
  \item a style of \emph{coordination language}: separate communication from
    computation
  \item tuple space is \emph{bag} of tagged data
  \item tagged data is added, examined and removed by concurrent processes (with
    locking)
  \item tuples may represent \emph{tasks} and \emph{results}
  \item eg \emph{Linda}, \emph{Chemical Abstract Machine},
    \emph{JavaSpaces}, \emph{Space-Based Architecture}\ldots
  \end{itemize}
\end{slide}

\begin{slide}
  \Heading{Peer-to-peer}
  \begin{itemize}
  \item clients and servers coincide: \emph{servents}
  \item typically an expectation that users of services will reciprocate
  \item minimize centralization, eg of registry, broker or router
  \item how do servents locate each other? perhaps initial use of
    registry, then boot-strapping (`hybrid p2p')
  \item reduces bottlenecks, increases robustness
  \item \emph{unstructured} vs \emph{structured} overlay networks:
    arbitrary or deterministic allocation of resources?
  \item eg file-sharing architectures such as Napster, Gnutella, BitTorrent
  \end{itemize}
\end{slide}

\begin{slide}
\Subheading{Agent-based}
\begin{itemize}
\item autonomous components: no central control
\item reasoning about environment and interactions
\item belief vs knowledge: modal and epistemic logic
\item devolved responsibility: action on behalf of user
\item maybe mobile
\end{itemize}
\end{slide}

\begin{slide}
\Heading{Interpreter}
\begin{itemize}
\item virtual machine in software
\item pseudo-program \emph{interpreted}
\item multiple levels of `state': interpreting and interpreted
\item eg \emph{rule-based system}
\begin{itemize}
\item fixed body of \emph{rules}
\item varying knowledge base of \emph{facts}
\item interpretation of rules by \emph{inference engine}
\end{itemize}
\end{itemize}
\end{slide}

\begin{slide}
\Heading{Grid/Cloud/Ubicomp\ldots}
\begin{itemize}
\item massively parallel, loosely coupled
\item ready access to shared resources \\
  (processor, space, equipment, expertise\ldots)
\item pay-per-use, on-demand
\item ``autonomic'' (self-managed)
\end{itemize}
\end{slide}

\begin{slide}
\Subheading{Grid computing}
\begin{itemize}
\item by analogy with electricity grid
\item pioneered by SETI@Home
\begin{quote}
3 million volunteers downloading and analyzing radio telescope data
\end{quote}
\item loosely coupled, heterogeneous, geographically dispersed
\item \emph{virtual organizations}
\item move computation to data rather than vice versa (eg LHC)
\end{itemize}
\end{slide}

\begin{slide}
\Subheading{Grid services}
\begin{itemize}
\item web services for grid computing
\item stateful, by necessity (now exploiting WSRF)
\item \textsc{Factory} pattern for service instantiation
\item see Open Grid Services Architecture (OGSA) \smallskip\\
\small
\url{http://www.globus.org/alliance/publications/papers/ogsa.pdf}
\end{itemize}
\end{slide}

\begin{slide}
\Subheading{climate\textit{prediction}.net}
\begin{minipage}{0.5\linewidth}\raggedright
\begin{itemize}
\item users donate computer time
\item in the style of \texttt{SETI@home}
\item but jobs run for weeks, not hours
\item running a climate model from the Met. Office
\item upload 10Mb of data to servers around the world
\item build probabilistic model to predict where we'll be in 2050
\end{itemize}
\end{minipage}
\qquad
\begin{minipage}{0.4\linewidth}
\begin{flushright}
\includegraphics[width=\linewidth]{diagrams/cpdn}
\end{flushright}
\end{minipage}\hfill
\end{slide}

\begin{slide}
\Subheading{eDiamond}
\begin{minipage}{0.5\linewidth}\raggedright
\begin{itemize}
\item national database of mammographic images
\item applications in training and diagnosis
\item Oxford, IBM, various hospitals
\item 12MB per image \ldots
\item major requirements in security, data location,
	access times
\end{itemize}
\end{minipage}
\qquad
\begin{minipage}{0.4\linewidth}
\begin{flushleft}
\includegraphics[width=\linewidth]{diagrams/ediam1}
\bigskip \\
\includegraphics[width=\linewidth]{diagrams/ediam2}
\end{flushleft}
\end{minipage}
\end{slide}

\begin{slide}
\Subheading{Cloud computing}
\begin{itemize}
\item ``software as a service'', rented not bought
\item services and data hosted online, available anywhere
\item data centres, virtualization technologies
\item `The Cloud' is a metaphor for the internet (eg in network diagrams)
\item eg Amazon WS, Google Apps, Microsoft Azure
\end{itemize}
\end{slide}

\begin{slide}
\Subheading{Map-reduce}
\begin{itemize}
\item Google's programming model for cluster computing
\item inspired by functional programming
\item works on sets of key/value pairs
\item \emph{map} over each pair a function giving intermediate k/v pairs
\item for each key, \emph{reduce} all associated values to some summary
\item framework handles execution on commodity hardware: partitioning, scheduling, communication, fault-tolerance
\item many examples: histograms (wordcount, access stats), distributed grep and sort, reverse web links, inverted index\ldots
\item open-source implementation as Apache Hadoop
\end{itemize}
\end{slide}

\begin{slide}
\Subheading{Ubiquitous computing}
\begin{itemize}
\item information processing in everyday objects
\item handheld devices, consumer goods, sensors, RFID\ldots
\item ad-hoc networks, innovative UIs
\item ``pervasive computing'', ``Internet of Things'', ``everyware''
\item see MOB
\end{itemize}
\end{slide}

\begin{slide}
  \Listofslides
\end{slide}

\begin{slide}
  \Timetable
\end{slide}

\end{document}
