\newif\iffaked \fakedfalse
\iffaked
\documentclass{sepslide-soa-faked} % jim has mucked up xdvi settings
\else
\documentclass{sepslide-soa}
\fi

\title{Engineering SOA}
\topic{9}{Engineering SOA}[BlueGreen]
\indexfile{00-index.pdf}

\begin{document}

\begin{slide}
  \Title
\end{slide}

\begin{slide}
  \Contents
\end{slide}

\begin{slide}
\Heading{Organization}
\begin{itemize}
\item SOA is not just a technical development
\item successful adoption has organizational consequences too
\bigskip
\item arguments from Josuttis, \textit{SOA in Practice}
\end{itemize}
\end{slide}

\begin{slide}
\Subheading{Before SOA: silos}
\begin{flushleft}
\includegraphics[height=0.8\textheight]{diagrams/silos}
\end{flushleft}
Separate fiefdoms, controlling independent systems.
\end{slide}

\begin{slide}
\Subheading{With SOA: services}
\begin{flushleft}
\includegraphics[height=0.8\textheight]{diagrams/layered-abstraction}
\end{flushleft}
\end{slide}

\begin{slide}
\Subheading{Organizational structure}
\begin{itemize}
\item refactoring of fiefdoms:
  \begin{itemize}
  \item backend departments
  \item cross-domain departments
  \item frontend departments
  \item ``solutions managers'' 
  \end{itemize}
\item requires collaboration and trust
\end{itemize}
\end{slide}

\begin{slide}
\Subheading{Conway's law}
\begin{quote} 
{\sffamily Any organization that designs a system will
    inevitably produce a design whose structure is a copy of the
    organization's communication structure.
} \medskip \\
(Melvin Conway, \textit{How Do Committees Invent?}, Datamation Apr 1968,
\url{http://www.melconway.com/law/})
\end{quote}

Popularized and named by Fred Brooks in \textit{The Mythical Man-Month}:
``If you have four groups working on a compiler, you'll get a 4-pass compiler.''

Conversely\ldots
\end{slide}

\begin{slide}
\Heading{Lifecycle}
\begin{itemize}
\item services are software
\item standard software lifecycle practices
\item but services also part of larger organizational processes
\item therefore some differences
\end{itemize}
\end{slide}

\begin{slide}
\Subheading{Core phases}
\begin{center}
\includegraphics[height=0.8\textheight]{diagrams/service-lifecycle-core}
\end{center}
\end{slide}

\begin{slide}
\Subheading{Iterative development in context}
\begin{center}
\includegraphics[height=0.8\textheight]{diagrams/service-lifecycle-iterate}
\end{center}
Some decisions will need revisiting.
\end{slide}

\begin{slide}
\Subheading{Identifying services}
\begin{center}
\includegraphics[height=0.8\textheight]{diagrams/service-lifecycle-identify}
\end{center}
Requirements, analysis, portfolio management\ldots
\end{slide}

\begin{slide}
\Subheading{Fixing services}
\begin{center}
\includegraphics[height=0.8\textheight]{diagrams/service-lifecycle-fix}
\end{center}
If it's just a bug-fix, fix it in-place.
\end{slide}

\begin{slide}
\Subheading{Modifying services}
\begin{center}
\includegraphics[height=0.8\textheight]{diagrams/service-lifecycle-modify}
\end{center}
If it's in any way significant, make a new service.
\end{slide}

\begin{slide}
\Subheading{Withdrawing services}
\begin{center}
\includegraphics[height=0.8\textheight]{diagrams/service-lifecycle-withdraw}
\end{center}
deprecate $\to$ monitor $\to$ persuade $\to$ delete
\end{slide}

\begin{slide}
\Heading{Versioning}
\begin{itemize}
\item waterfall principles giving way to agile
\item SOA is designed for flexibility
\item especially for large systems, cannot fix requirements
\medskip
\item \emph{but still} people want interfaces to be stable
\item contracts and projects depend on it
\item not feasible to refactor large systems
\item so stability is also very important
\end{itemize}
\end{slide}

\begin{slide}
\Subheading{Simple domain-driven versioning}
\begin{itemize}
\item there are complex approaches to versioning, with automatic updates
\item simpler approach often suffices:
\begin{quote}
\emph{every published modification yields a new service}
\end{quote}
\item make version number part of the service identifier:
$GetCustomerData1$, $GetCustomerData2$
\item (or version number might be a parameter, or contextual)
\end{itemize}
\end{slide}

\begin{slide}
\Subheading{Backwards compatibility}
\begin{itemize}
\item relax the rule for bug-fixes
\item encourage compliance for ``backwards-compatible'' changes
  \begin{itemize}
  \item might not be so compatible, eg if SLA changes
  \item maybe changes in datatypes\ldots
  \item besides, any bug-fix entails some risk
  \end{itemize}
\item require compliance for incompatible changes
\medskip
\item actually, bug-fixes might be (even incompatible) changes
\end{itemize}
\end{slide}

\begin{slide}
\Subheading{Versioning of datatypes}
\begin{itemize}
\item modified services often require modified datatypes
\item these differences propagate through aggregation
\item sooner or later, consumers will have to deal with multiple versions of `the same' datatype
\item three approaches:
\begin{itemize}
\item really use different types 
\item use one common type, with optional values
\item don't use types, make it generic
\end{itemize}
\end{itemize}
\end{slide}

\begin{slide}
\Subheading{Propagation of datatype versions}
\begin{flushleft}
\includegraphics[height=0.8\textheight]{diagrams/versioning}
\end{flushleft}
\end{slide}

\begin{slide}
\Heading{Governance}
\begin{itemize}
\item SOA has technical aspects
\item but it isn't a product: you can't buy it
\item how can you introduce it?
\item how can you govern its application?
\end{itemize}
\end{slide}

\begin{slide}
\Subheading{Non-technical governance issues}
\begin{itemize}
\item \emph{visions, objectives, business case, funding model}
\begin{quote}
why are we doing this? how will we pay for it?
\end{quote}
\item \emph{reference architecture}
\begin{quote}
fundamental decisions: preferred technology, message exchange patterns, metamodel, etc
\end{quote}
\item \emph{rules and responsibilities}
\begin{quote}
who drives and cares about issues
\end{quote}
\item \emph{policies, standards, formats, processes, lifecycles}
\begin{quote}
decide and document, in standard notations
\end{quote}
\end{itemize}
\end{slide}

\begin{slide}
\Subheading{Technical governance issues}
\begin{itemize}
\item \emph{documentation}
\begin{quote}
important for transparency; promotes non-technical issues
\end{quote}
\item \emph{service management}
\begin{quote}
repositories and registries for services and contracts
\end{quote}
\item \emph{monitoring}
\begin{quote}
conformance to policies, meeting SLAs, preparing for withdrawal
\end{quote}
\item \emph{change and configuration management}
\begin{quote}
the usual tools 
\end{quote}
\end{itemize}
\end{slide}

\begin{slide}
\Subheading{SOA step-by-step}
\begin{itemize}
\item \emph{understand it}
\begin{quote}
what's the benefit? do you have support?
\end{quote}
\item \emph{carry out a pilot project}
\begin{quote}
try it out, small but not too small; should have some value
\end{quote}
\item \emph{second and third projects}
\begin{quote}
determine what was specific to pilot; 
refactor these reference implementations
\end{quote}
\item \emph{develop a general strategy}
\begin{quote}
establish useful lessons and principles
\end{quote}
\end{itemize}
Don't forget documentation and retrospective reviews, at every step.
\end{slide}

\begin{slide}
\Subheading{Establishing SOA: four approaches}
\begin{itemize}
\item \emph{developer-driven}, grass-roots
\begin{quote}
leads to technological experience; 
likely to be uncoordinated
\end{quote}
\item \emph{business-driven}
\begin{quote}
proof of concept helps adoption;
limited benefit from early projects
\end{quote}
\item \emph{IT-driven}
\begin{quote}
effective for infrastructure;
focus on technical aspects
\end{quote}
\item \emph{management-driven}, top-down
\begin{quote}
coordinated, driven by business priorities;
expensive, disruptive, risky
\end{quote}
\end{itemize}
Really need a combination of them all.
\end{slide}

\begin{slide}
  \Listofslides
\end{slide}

\begin{slide}
  \Timetable
\end{slide}

\end{document}
