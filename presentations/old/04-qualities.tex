\newif\iffaked \fakedfalse
\iffaked
\documentclass{sepslide-soa-faked} % jim has mucked up xdvi settings
\def\colourbullet#1{\item[#1]}
\else
\documentclass{sepslide-soa}
\def\colourbullet#1{\item[\large\textbf{\textcolor{BulletColor}{#1}}]}
\fi
\def\pro{\colourbullet{(+)}}
\def\con{\colourbullet{(--)}}

\title{Service qualities}
\topic{4}{Service qualities}[ForestGreen]
\indexfile{00-index.pdf}

\begin{document}

\begin{slide}
  \Title
\end{slide}

\begin{slide}
  \Contents
\end{slide}

\begin{slide}
\Heading{Transactions}
\begin{itemize}
\item idealized \emph{indivisible activities}
\item techniques for maintaining the illusion in the face of complexity, concurrency, failures
\item ideas arose from distributed databases
\item underlying finance, logistics, manufacturing\ldots
\item \textit{Transaction Processing: Concepts and Techniques}, Gray and Reuter, 1993 (\url{http://books.google.co.uk/books?id=S_yHERPRZScC})
% \item \url{http://research.microsoft.com/~gray/WICS_99_TP/}
\end{itemize}
\end{slide}

\begin{slide}
\Subheading{ACID properties}
\begin{itemize}
\item \emph{atomicity}
\begin{quote}
all-or-nothing
\end{quote}
\item \emph{consistency}
\begin{quote}
integrity-preserving: invariants satisfied
\end{quote}
\item \emph{isolation}
\begin{quote}
hidden intermediate results:
multi-user behaviour consistent with single-user mode
\end{quote}
\item \emph{durability}
\begin{quote}
permanent committed results
\end{quote}
\end{itemize}
\end{slide}

\begin{slide}
\Subheading{Problems avoided}
\begin{itemize}
\item \emph{lost update}
\begin{quote}
write committed and acknowledged but then discarded
\end{quote}
\item \emph{inconsistent retrieval}
\begin{quote}
reads of multiple fields at different times
\end{quote}
\item \emph{non-serializability}
\begin{quote}
loss of single-user abstraction
\end{quote}
\item \emph{conflict}
\begin{quote}
eg simultaneous bookings of the same room
\end{quote}
\end{itemize}
\end{slide}

\begin{slide}
\Subheading{Transaction lifecycle}
\begin{quote}
\includegraphics[width=\linewidth]{diagrams/transaction}
\end{quote}
\end{slide}

\begin{slide}
\Subheading{Two-phase commit}
\begin{itemize}
\item \emph{initiator} for each transaction, but any participant may abort 
\item 2PC protocol minimizes unavailability of unilateral abort 
\item \emph{commit-request} phase
\begin{quote}
initiator sends $commit?$ message to all participants,
who vote either $yes!$ or $no!$
\end{quote}
\item \emph{commit} phase
\begin{quote}
initiator sends $commit!$ (if unanimously $yes!$)
or $abort!$ (otherwise)
to all participants, who act and $acknowledge$;
then initiator completes transaction
\end{quote}
\item some disadvantages (blocking during 2PC; central control) so many variants\ldots
\item cf Christian wedding ceremony
\end{itemize}
\end{slide}

\begin{slide}
\Subheading{Locking}
\begin{itemize}
\item \emph{serialization mechanisms} for resources, enforcing unique access
\item \emph{read locks} (shareable) and \emph{write locks} (exclusive)
\item may need to wait for locked resource to be released
\item may result in \emph{deadlock}: two parties, each waiting for the other % to release a lock
\item lock resources in canonical order (requires foresight), or abort one party (requires rollback)
\item chance of conflict rises as square of degree of multiplicity, and fourth power of size of transaction
\item locks of varying granularity: arrange into DAG, lock from root to leaf and release in opposite order
  (coarser locks reduce overhead, but increase contention)
\item see Gray, \textit{Notes on DB OSes}, DOI \url{10.1007/3-540-08755-9_9}
\end{itemize}
\end{slide}

\begin{slide}
\Subheading{Nested transactions}
\begin{itemize}
\item expensive transaction fails part-way through
\item abort wastes much useful work
\item how to avoid this, while staying acidic?
\item organize transaction into tree of sub-transactions
\item when sub-transaction commits, results visible only to parent
\item if transaction rolls back, so do all its sub-transactions
\item note: sub-transactions only ACI, not D
\end{itemize}
\end{slide}

\begin{slide}
\Subheading{Long-running transactions}
\begin{itemize}
\item lasting days, not seconds: \emph{sagas}
\item too expensive to holds locks,
   or they're not available (eg for human participants)
\item optimism+compensation rather than pessimism+roll-back
\item compensation usually only approximates `undo' \\
  (cancellation fee, overstocking,\ldots)
\item misnomer: neither A nor I, and C must take non-I into account
\end{itemize}
\end{slide}

\begin{slide}
\Subheading{WS-Transaction (WSTX)}
\begin{itemize}
\item OASIS standard
\item \emph{WS-Coordination} for managing conversational context
\item \emph{WS-AtomicTransaction} for short-lived activites with full ACID properties (two-phase commit)
\item \emph{WS-BusinessActivity} for long-running transactions with compensation
\end{itemize}
\end{slide}

%----------------------------------------------------------------------

\begin{slide}
\Heading{Performance}
\begin{itemize}
\item two common reasons for systems not to live up to expectations:
  performance and security
\item SOA about heterogeneity, so translations necessary
\item SOA about distribution, so network latency an issue too
\item how much do these matter?
\end{itemize}
(material from Josuttis, \textit{SOA in Practice})
\end{slide}

\begin{slide}
\Subheading{Lifecycle of a service call}
\begin{flushleft}
\includegraphics[height=0.8\textheight]{diagrams/call-sequence}
\end{flushleft}
\end{slide}

\begin{slide}
\Subheading{Serializing and deserializing}
\begin{itemize}
\item necessary: cannot assume a common representation
\item formatting, parsing and validation for XML messages take time
\item increased bandwidth from chatty presentation (factor of 4 to 20 growth)
\item message-level security aspects have an effect too
\end{itemize}
\end{slide}

\begin{slide}
\Subheading{Transmission}
\begin{itemize}
\item in-memory call overhead about 100ns, service request about 100ms
\item tempting to ignore network latency
\item only possible for coarse-grained interactions
\item service interface should provide convenient batched/chunked access
\item (still awkward for client to use, though)
\item treat automated OO--WS mappings with caution!
\end{itemize}
\end{slide}

\begin{slide}
\Subsubheading{Example: \textsc{Iterator}}
\begin{itemize}
  \item Java $Vector$ yields elements one at a time:
  \begin{quote}
  \includegraphics[scale=0.75]{diagrams/iteration-java.eps}
  \end{quote}
  \item typical distributed service is more careful~--- eg CORBA:
  \begin{quote}
  \includegraphics[scale=0.75]{diagrams/iteration-corba.eps}
  \end{quote}
\end{itemize}
\end{slide}

\begin{slide}
\Subsubheading{Example: \textsc{Remote Fa\c cade} (Fowler, PoEAA)}
\begin{itemize}
\item \emph{provides a coarse-grained fa\c cade on fine-grained objects to improve efficiency over a network}
  \medskip
\item batched access to small methods
\item fine-grained objects have no remote interface; fa\c cade has no business logic
\item more obscure, more awkward: costs wrt productivity
\end{itemize}
  \begin{quote}
  \includegraphics[scale=0.75]{diagrams/remote-facade.eps}
  \end{quote}
% (Martin Fowler, \textit{Patterns of Enterprise Application Architecture})
\end{slide}

\begin{slide}
\Subsubheading{Example: \textsc{Data Transfer Object} (Fowler, PoEAA)}
\begin{itemize}
\item \emph{an object that carries data between processes in order to reduce the number of method calls}
%  \medskip
%\item with \textsc{Remote Fa\c cade}, reduce number of calls, but pack more data into each call
%\item could use lots of parameters and results---awkward, sometimes impossible
%\item instead, make a composite DTO
%\item serialize for transmission
%\item usually an assembler/disassembler on server side
\end{itemize}
  \begin{quote}
  \includegraphics[scale=0.6]{diagrams/data-transfer-object.eps}
  \end{quote}
% (Martin Fowler, \textit{Patterns of Enterprise Application Architecture})
\end{slide}

\begin{slide}
\Subheading{Distribution can't be transparent}
% "A Note on Distributed Computing"
% Jim Waldo, Geoff Wyant, Ann Wollrath and Sam Kendall, 
% SMLI TR-94-29, Sun Microsystems Labs Inc, Nov 1994
\begin{quote} \itshape
The hard problems in distributed computing concern dealing with
partial failure and the lack of a central resource manager\ldots
differences in memory access paradigms between local and distributed
entities.

Many aspects of robustness can be reflected only at the
protocol/interface level\ldots An interface design issue has put an
upper bound on the performance\ldots Part of the definition of a class
of objects will be the specification of whether those objects are
meant to be used locally or remotely.
    \begin{flushright} \upshape
    Jim Waldo et al, `A Note on Distributed Computing', 1994
    \end{flushright}
\end{quote}
\end{slide}

\begin{slide}
\Subheading{Processing}
\begin{itemize}
\item major reason for slow service is processing time
\item particularly an issue in the face of scaling
\item service contract should \emph{service level agreement} (SLA), specifying average response time and call rate
\item might have to replicate service to achieve throughput
\item if that doesn't work, need to decouple consumer (eg batching, asynchrony)
\item as with all optimizations, worth profiling first!
\end{itemize}
\end{slide}

\begin{slide}
\Subheading{Performance and reusability}
\begin{itemize}
\item trade-off between conflicting forces
\item latency encourages \emph{coarse-grained} services 
(send one request rather than several)
\item processing time encourages \emph{fine-grained} services
(don't send data that isn't needed)
\item Josuttis proposes introducing special service for awkward consumer who wants an odd selection of fields
(so much for reuse!)
\item could have generic parametrized service, passed a \emph{strategy}; but that has processing implications of its own
\end{itemize}
\end{slide}

\begin{slide}
\Subheading{Performance and backwards compatibility}
\begin{itemize}
\item another Josuttis war story: CRM system for phone company
\item calls routed to operator; customer data preloaded to screen
\item now company introduces an extra $status$ attribute of customer
(high status customers get routed to more experienced operators)
\item simply another field returned by service: backwards compatible
\medskip
\item customer calls from one number, but may have multiple contracts
\item so determine status from \emph{total} volume of contracts
\item but now have to load and process all contracts at run-time! doubled response time
\item (solution: to compute status overnight; still expensive)
\item moral: SLA is part of the contract too
\end{itemize}
\end{slide}

%----------------------------------------------------------------------

\begin{slide}
	\Heading{Security}
Classical notions of security are very relevant to SOA:
	\begin{itemize}
\item \emph{authentication} (who are you?)
\item \emph{authorization} (what may you do?)
\item \emph{confidentiality} (should you see this?)
\item \emph{integrity} (may you change this?)
\item \emph{availability} (are you being fair?)
\item \emph{accountability} (who is paying?)
\item \emph{auditability} (what happened?)
	\end{itemize}
	--- few services will have no security requirements at all.
\end{slide}

\begin{slide}
	\Subheading{\textit{Ad hoc} approaches to security}
	\begin{itemize}
	\item just don't think about it
	\item just don't tell anyone your service is there
	\item just use firewalls to limit the service to the 
		organisational intranet
	\item just roll your own
	\end{itemize}
What do you reckon?
\end{slide}

\begin{slide}
	\Subheading{Means of security}
	\begin{itemize}
	\item identification, authentication schemes
	\item authorization policies and services
	\item cryptography: encryption and signatures
	\item logging, logging services
	\item policy management
	\end{itemize}
\end{slide}

\begin{slide}
	\Subheading{Basic technologies}
	\begin{itemize}
	\item PKI (Public-Key Infrastructure)
	\item simple tokens
	\item cookies
	\item HTTP authentication
	\item SSL (HTTPS); also IPSec
	\item XML Encryption, XML Signature
	\item SOAP Signatures
	\end{itemize}
Contrast \emph{message-layer security} with \emph{transport-layer security}.
\end{slide}

\begin{slide}
	\Subheading{PKI in a nutshell}
	\begin{itemize}
	\item public-key encryption uses \emph{key pairs}
	\item the encryption is asymmetric: one key \emph{encrypts},
		the other \emph{decrypts}
	\item one key published: anyone can encrypt messages for me
	\item other key private: only I can decrypt
	\item signatures are dual: only I can sign, anyone can verify
	\item infeasible to reconstruct private key from public
	\colourbullet{Q:} how do I verify that a key really belongs to a particular individual?
	\colourbullet{A:} use a certificate that binds a public key to a name, via a signature from someone I already trust.
	\end{itemize}
\end{slide}

\begin{slide}
	\Subheading{Simple tokens}
	\begin{itemize}
	\item used by Google and Amazon Web Service APIs
	\item \xmlemph{doGoogleSearch(\emph{key}, \ldots)}
	\item token is set up out-of-band
	\item well-suited to a REST-style application
	\pro easy to set up
	\pro meets the service-provider's security goals 
			(log/audit, availability management)
	\con no protection against impersonation
	\end{itemize}
\end{slide}

\begin{slide}
	\Subheading{HTTP authentication}
	\begin{itemize}
	\item \textbf{`basic version'}: username and password are base64
	encoded and passed in an HTTP header
	\pro implemented as standard
	\con not really much better than the simple token
\medskip
	\item \textbf{`digest version'}: server sends a \emph{nonce challenge};
		response incorporates a digest value (MD5 hashes of
		nonce, username, password, other message data)
        \pro much better; exposure is limited
	\con server must choose good nonces to prevent replay attacks;
		difficult to manage in `session' contexts
	\con protects secrecy of password, but offers no other confidentiality
	\end{itemize}
\end{slide}

\begin{slide}
	\Subheading{SSL/HTTPS}
	\begin{itemize}
	\item strong authentication of server; 
		optional strong authentication of client
	\item strongly encrypted two-way data channel
	\pro standard and straightforward
	\pro highly trustworthy
	\con entirely point-to-point
	\con no scope for processing by intermediaries
	        (that includes firewalls and filters)
	\con protocol-based; no scope for fine-grained authorization
	\end{itemize}
\end{slide}

\begin{slide}
	\Subheading{XML encryption}
	\begin{itemize}
	\item use \xmlemph{<EncryptedData>}, \xmlemph{<CipherData>}
		to signal and contain the encrypted (base-64) data
	\item other tags record choice of algorithms, identity of key, etc
	\item very flexible: decide precisely what to encrypt; 
		incorporate in your schema!
        \item good reading: \textit{Why XML Security is Broken} by Peter Gutmann
	\end{itemize}
\end{slide}

\begin{slide}
\Subheading{Example}
\begin{xml}
<?xml version='1.0'?>
  <PaymentInfo xmlns='http://example.org/paymentv2'>
    <Name>John Smith</Name>
    <CreditCard Limit='5,000' Currency='USD'>
      \xmlemph{<Number>4019 2445 0277 5567</Number>}
      <Issuer>Example Bank</Issuer>
      <Expiration>04/02</Expiration>
    </CreditCard>
  </PaymentInfo>
\end{xml}
(Example from XML Encryption Standard, W3C, 10 Dec 2002)
\end{slide}

\begin{slide}
\Subsubheading{Encrypt the \xmlfrag{CreditCard} element}
\begin{xml}
  <?xml version='1.0'?>
  <PaymentInfo xmlns='http://example.org/paymentv2'>
    <Name>John Smith</Name>
    \xmlemph{<EncryptedData Type='http://www.w3.org/2001/04/xmlenc#Element'
                           xmlns='http://www.w3.org/2001/04/xmlenc#'>
      <CipherData>
        <CipherValue>A23B45C56</CipherValue>
      </CipherData>
    </EncryptedData>}
  </PaymentInfo>
\end{xml}
(Eavesdropper can't even distinguish credit card use from money transfer.)
\end{slide}

\begin{slide}
\Subsubheading{Encrypt finer-grained elements}
\begin{xml}
  <?xml version='1.0'?> 
  <PaymentInfo xmlns='http://example.org/paymentv2'>
    <Name>John Smith</Name>
    \xmlemph{<CreditCard Limit='5,000' Currency='USD'>}
      <EncryptedData xmlns='http://www.w3.org/2001/04/xmlenc#'
                      Type='http://www.w3.org/2001/04/xmlenc#Content'>
        <CipherData>
          <CipherValue>A23B45C56</CipherValue>
        </CipherData>
      </EncryptedData>
    \xmlemph{</CreditCard>}
  </PaymentInfo>
\end{xml}
Credit card limit in the clear, but other elements (number, issuer, expiration) encrypted.
\end{slide}

\begin{slide}
\Subsubheading{Encrypt the content}
\begin{xml}
  <?xml version='1.0'?> 
  <PaymentInfo xmlns='http://example.org/paymentv2'>
    <Name>John Smith</Name>
    \xmlemph{<CreditCard Limit='5,000' Currency='USD'>
      <Number>}
        <EncryptedData xmlns='http://www.w3.org/2001/04/xmlenc#'
          Type='http://www.w3.org/2001/04/xmlenc#Content'>
          <CipherData>
            <CipherValue>A23B45C56</CipherValue>
          </CipherData>
        </EncryptedData>
      \xmlemph{</Number>
      <Issuer .../> <Expiration .../>
    </CreditCard>}
  </PaymentInfo>
\end{xml}
Credit card and \xmlfrag{Number} tag in the clear, but number itself encrypted.
\end{slide}

\begin{slide}
\Subsubheading{Or encrypt the whole document}
\begin{xml}
  <?xml version='1.0'?> 
  <EncryptedData xmlns='http://www.w3.org/2001/04/xmlenc#'
   MimeType='text/xml'>
    \xmlemph{<CipherData>
      <CipherValue>A23B45C56</CipherValue>
    </CipherData>}
  </EncryptedData>
\end{xml}
\end{slide}

\begin{slide}
\Subsubheading{Additional tags}
\begin{xml}
<EncryptedData xmlns='http://www.w3.org/2001/04/xmlenc#'
        Type='http://www.w3.org/2001/04/xmlenc#Element'/>
   \xmlemph{<EncryptionMethod
          Algorithm='http://www.w3.org/2001/04/xmlenc#tripledes-cbc'/>
   <ds:KeyInfo xmlns:ds='http://www.w3.org/2000/09/xmldsig#'>
     <ds:KeyName>John Smith</ds:KeyName>
   </ds:KeyInfo>}
   <CipherData><CipherValue>DEADBEEF</CipherValue></CipherData>
</EncryptedData>
\end{xml}
\end{slide}

\begin{slide}
	\Subheading{XML Signature}
	\begin{itemize}
	\item more complex than encryption:
		\begin{itemize}
		\item intimately tied to representation
		\item \emph{canonicalization} needed
		\end{itemize}
	\item tampering with document, followed by removal of signature,
		 may be invisible
	\item so design should take account of what to do with unsigned documents
	\item \emph{detached signatures} used to allow whole 
		document to be signed
	\end{itemize}
\end{slide}

\begin{slide}
\Subheading{Example of detached signature}
\begin{xml}
 <Signature Id="MyFirstSignature"
      xmlns="http://www.w3.org/2000/09/xmldsig\#"> 
  <SignedInfo> 
   <CanonicalizationMethod
       Algorithm="http://www.w3.org/TR/2001/REC-xml-c14n-20010315"/> 
   <SignatureMethod
       Algorithm="http://www.w3.org/2000/09/xmldsig\#dsa-sha1"/> 
   <Reference
       URI="http://www.w3.org/TR/2000/REC-xhtml1-20000126/"> 
     <Transforms> <Transform 
         Algorithm="http://www.w3.org/TR/2001/REC-xml-c14n-20010315"/>
     </Transforms> 
     <DigestMethod Algorithm="http://www.w3.org/2000/09/xmldsig\#sha1"/> 
     <DigestValue>\xmlemph{j6lwx3rvEPO0vKtMup4NbeVu8nk=}</DigestValue> 
   </Reference> 
  </SignedInfo> 
\end{xml}\end{slide}
\begin{slide}\begin{xml}
  <SignatureValue>\xmlemph{MC0CFFrVLtRlk=...}</SignatureValue> 
  <KeyInfo> 
    <KeyValue>
      <DSAKeyValue> 
        <P>...</P><Q>...</Q><G>...</G><Y>...</Y> 
      </DSAKeyValue> 
    </KeyValue> 
  </KeyInfo> 
</Signature>
\end{xml}
(From \textit{XML Signature}, W3C Recommendation 12 February 2002)
\end{slide}

\begin{slide}
\Subheading{Web Services Security Roadmap}
\fboxsep=3bp
\def\block#1#2{\fbox{\hbox to #1{\hfil#2\strut\hfil}}}
\begin{center}
\mbox{\hbox to 316bp{%
\block{120bp}{WS-SecureConversation}\hfil
\block{72bp}{WS-Federation}\hfil
\block{100bp}{WS-Authorization}%
}} \\
\mbox{\hbox to 316bp{%
\block{120bp}{WS-Policy}\hfil
\block{72bp}{WS-Trust}\hfil
\block{100bp}{WS-Privacy}%
}} \\
\block{310bp}{WS-Security} \\ % 4*72bp-2*fboxsep
\block{310bp}{SOAP Foundation}
\end{center}
From IBM and Microsoft; see
\url{www.ibm.com/developerworks/library/specification/ws-secmap/}
\end{slide}

\begin{slide}
	\Subheading{Documents}
(Varying levels of maturity\ldots)
	\begin{description} 

	\item[WS-Security:] how to attach signature and encryption
	headers to SOAP messages; how to attach security tokens
	to messages 
(OASIS~standard)

	\item[WS-Policy:] capabilities and constraints of the security
	(and other business) policies on intermediaries and endpoints
(OASIS~submission)

	\item[WS-Trust:] framework for trust models that enables Web
	services to interoperate securely; introduces notion of a
	\emph{Security Token Service}
(OASIS~standard)

	\item[WS-Privacy:] model for how Web services and requesters
	state privacy preferences and organisational privacy practice
	statements
(vapourware)
	\end{description}
\end{slide}
\begin{slide}
	\begin{description}

	\item[WS-SecureConversation:] how to manage and authenticate
	message exchanges between parties, including security context
	exchange and establishing and deriving session keys~---
	long-running interactions, in contrast to those enabled by
	WS-Security 
(OASIS~standard)

	\item[WS-Federation:]
	how to manage and broker the trust relationships in a
	heterogeneous federated environment including support for
	federated identities
(just~a~draft)

	\item[WS-Authorization:] how to manage authorization data and
	authorization policies
(vapourware)

	\end{description}
	The roadmap also includes a number of instructive scenarios.
\end{slide}

\begin{slide}
	\Subheading{Terminology}
	\begin{itemize}
	\item security \emph{token}, signed or unsigned
	\item \emph{subject}: person, application, activity 
	\item \emph{claim}: made about the subject by the subject or another party

	\item \emph{proof-of-possession} of token or set of claims
	\item \emph{web service endpoint policy} for required claims
	% \item claim requirements --- policy applied to messages
	\item \emph{actor}: intermediary or endpoint (URI)
			(not user or client software)
	\end{itemize}

\end{slide}

\begin{slide}
\Subheading{Web Services trust model}

	\begin{itemize} 
	\item service may have a \emph{policy}, requiring
	requester to prove set of claims (e.g., name, key,
	permission, capability, etc.)

	\item requester can prove required
	claims by associating security tokens with messages

	\item when requester cannot prove required claims directly, it
	(or its agent) can try to do so by
        by contacting other \emph{security token services}~---
 which may in turn have their own policies

	\item security token services \emph{broker trust} between
          different trust domains by issuing security tokens
\end{itemize}
\end{slide}
\begin{slide}
\begin{center}
	\includegraphics[width=\textwidth]{diagrams/ws-security-model}
\end{center}
\end{slide}

\begin{slide}
  \Subheading{Push vs pull}
\includegraphics[width=\textwidth]{diagrams/push-pull}
\end{slide}

\begin{slide}
  \Subheading{Distributed decision-making}
\begin{flushleft}
\includegraphics[height=0.8\textheight]{diagrams/pep-pdp}
\end{flushleft}
PEP asks PDP for a decision. 
\end{slide}

\begin{slide}
  \Subheading{Consequences}
\begin{itemize}
  \item `difficult' security decisions taken in one place
  \item resource can be in a separate domain from security service
  \item scalable in the number of users or resources
  \item push and pull models have different scalability characteristics
  \item scope for multiple security services, too (but inconsistency?)
\end{itemize}
\end{slide}

\begin{slide}
  \Listofslides
\end{slide}

\begin{slide}
  \Timetable
\end{slide}

\end{document}

\begin{slide}
   	\Subheading{WS-Security}
	\begin{itemize}
	\item name is unfortunate!
	\item describes a \emph{message security model}
	\item in particular, describes how to use 
		XML encryption and signing in SOAP

	\item ``The message receiver SHOULD reject a message with
	invalid signature, missing or inappropriate claims as it is an
	unauthorized (or malformed) message.''
	\end{itemize}
\end{slide}

\begin{slide}
	\Subheading{Signing in SOAP}
\begin{xml}
<?xml version="1.0" encoding="utf-8"?>
<S:Envelope xmlns:S="http://www.w3.org/2001/12/soap-envelope"
            \xmlemph{xmlns:ds="http://www.w3.org/2000/09/xmldsig#"}>
  <S:Header>
    <m:path xmlns:m="http://schemas.xmlsoap.org/rp/">
      <m:action>http://fabrikam123.com/getQuote</m:action>
      <m:to>http://fabrikam123.com/stocks</m:to>
      <m:id>uuid:84b9f5d0-33fb-4a81-b02b-5b760641c1d6</m:id>
    </m:path>
    \xmlemph{<wsse:Security
      xmlns:wsse="http://schemas.xmlsoap.org/ws/2002/04/secext">}
      <wsse:UsernameToken Id="\xmlemph{MyID}">
        <wsse:Username>Zoe</wsse:Username>
      </wsse:UsernameToken>
\end{xml}\end{slide}
\begin{slide}\begin{xml}
      <ds:Signature>
        <ds:SignedInfo>
          <ds:CanonicalizationMethod Algorithm =
            "http://www.w3.org/2001/10/xml-exc-c14n#"/>
          <ds:SignatureMethod Algorithm =
            "http://www.w3.org/2000/09/xmldsig#hmac-sha1"/>
          <ds:Reference URI="\xmlemph{#MsgBody}">
            <ds:DigestMethod Algorithm =
              "http://www.w3.org/2000/09/xmldsig#sha1"/>
            <ds:DigestValue>LyLsF0Pi4wPU...</ds:DigestValue>
          </ds:Reference>
        </ds:SignedInfo>
\end{xml}\end{slide}
\begin{slide}\begin{xml}
        <ds:SignatureValue>DJbchm5gK...</ds:SignatureValue>
        <ds:KeyInfo>
          <wsse:SecurityTokenReference>
            \xmlemph{<wsse:Reference URI="#MyID"/>}
          </wsse:SecurityTokenReference>
        </ds:KeyInfo>
      </ds:Signature>
    \xmlemph{</wsse:Security>}
  </S:Header>
  \xmlemph{<S:Body Id="MsgBody">}
    \xmlemph{<tru:StockSymbol xmlns:tru="http://fabrikam123.com/payloads">
        QQQ}
    \xmlemph{</tru:StockSymbol>}
  \xmlemph{</S:Body>}
</S:Envelope>
\end{xml}
\end{slide}

\begin{slide}
	\Subheading{Tag summary}
	\begin{itemize}
	\item \xmlfrag{Security} element in SOAP headers;  may be several,
		for different actors
	\item \xmlfrag{UsernameToken} records usernames and (optionally) passwords
	\item \xmlfrag{BinarySecurityToken} or \xmlfrag{SecurityTokenReference}
		provide for inclusion of/pointing to a security token
	\item \xmlfrag{Reference} of element being signed: could be in body or
		header
	\item \xmlfrag{ds:Signature} XML Signature element; may be
	cross-referenced with security tokens
	\end{itemize}
\end{slide}

\begin{slide}
\Subheading{Example: UsernameToken}
\begin{xml}
<S:Envelope 
  xmlns:S="http://www.w3.org/2001/12/soap-envelope"
  xmlns:wsse="http://schemas.xmlsoap.org/ws/2002/04/secext">
  <S:Header>
    ...
    <wsse:Security>
      <wsse:UsernameToken>
        <wsse:Username>Zoe</wsse:Username>
        <wsse:Password>ILoveDogs</wsse:Password>
      </wsse:UsernameToken>
    </wsse:Security>
    ...
  </S:Header>
  ...
</S:Envelope>
\end{xml}
\end{slide}

\begin{slide}
\Subheading{Example: BinarySecurityToken}
\begin{xml}
<wsse:BinarySecurityToken 
  xmlns:wsse="http://schemas.xmlsoap.org/ws/2002/04/secext" 
  Id="myToken"
  ValueType="x:MyType" xmlns:x="http://fabrikam123.com/x"
  EncodingType="wsse:Base64Binary">
  MIIEZzCCA9CgAwIBAgIQEmtJZc0...
</wsse:BinarySecurityToken>
\end{xml}
\end{slide}

%\begin{slide}
%	\Subheading{XML Encryption}
%	May be applied to header or body elements.  
%
%	Namespace: \uri{http://www.w3.org/2001/04/xmlenc\#}
%\end{slide}

%\begin{slide}
%	\Subheading{WS-Trust}
%	Standard messages for
%	\begin{itemize}
%	\item requesting security tokens
%	\item returning security tokens
%	\item token lifetime and renewal management
%	\item challenge-response protocols
%	\end{itemize}
%\end{slide}

\begin{slide}
	\Subheading{WS-Policy}
	A number of complementary / competing / overlapping
	standards consider how to
	\begin{itemize}
	\item describe policy
	\item reach decisions based on policies
	\item document credentials and decisions
	\item etc.
	\end{itemize}
	All in XML, naturally --- attach to messages, attach to
	service descriptions, etc. 
\bigskip \\
Good reference: \textit{Understanding WS-Policy}, Aaron Skonnard, August 2003.
\url{http://msdn.microsoft.com/en-us/library/ms996497.aspx}
\end{slide}

\begin{slide}	
	\Subheading{Taxonomy}
\begin{itemize}
\item \emph{Policy}: an informal abstraction; the set of information
  that is being expressed as policy assertions
\item \emph{Policy Assertion}: an individual preference,
requirement, capability or other property
\item \emph{Policy Expression}: an XML Infoset representation of one or
more policy assertions
\item \emph{Policy Subject}: an entity (e.g., an endpoint, object, or
resource) to which a policy expression can be bound
\item \emph{Policy Attachment}: the mechanism for associating policy
expressions with one or more subjects
\end{itemize}
For expressing policy; not necessarily giving an evaluation
or enforcement mechanism.
\end{slide}

\begin{slide}
\Subheading{Example from WS-SecurityPolicy}
\begin{xml}
<wsp:Policy xmlns:wsse="..." xmlns:wsp="...">
  <wsp:\xmlemph{ExactlyOne}>
    <wsse:SecurityToken wsp:Usage="wsp:Required" wsp:Preference="100">
      <wsse:TokenType>wsse:Kerberosv5TGT</wsse:TokenType>
    </wsse:SecurityToken>
    <wsse:SecurityToken wsp:Usage="wsp:Required" wsp:Preference="1">
      <wsse:TokenType>wsse:X509v3</wsse:TokenType>
    </wsse:SecurityToken>
  </wsp:ExactlyOne>
</wsp:Policy>
\end{xml}
\end{slide}

\begin{slide}
	\Subheading{WS-Policy operators}

	\xmlemph{<wsp:ExactlyOne>}
	\xmlemph{<wsp:OneOrMore>}

	\xmlemph{<wsp:All>}, equivalently \xmlemph{<wsp:Policy>}

	\Subheading{WS-Policy Usage Assertions}

	\xmlemph{<wsp:Required>} \xmlemph{<wsp:Rejected>}
	\xmlemph{<wsp:Optional>} \xmlemph{<wsp:Observed>}
	\xmlemph{<wsp:Ignored>}
	
\xmlemph{<wsp:Preference>} biases choice among alternatives.

Augmented by policy assertions from WS-PolicyAssertions


\end{slide}


\begin{slide}
	\Subheading{WS-SecurityPolicy}
	Extensions to WS-Policy in order to express assertions on:
	\begin{itemize}
	\item security tokens (\xmlemph{wsse:SecurityToken})
	\item integrity (\xmlemph{wsse:Integrity})
	\item confidentiality (\xmlemph{wsse:Confidentiality})
	\item visibility, message age, security headers etc
	\end{itemize}
\end{slide}

\begin{slide}
\Subheading{Example}
\begin{xml}
<wsse:Integrity wsp:Usage="wsp:Required">
  <wsse:Algorithm  Type="wsse:AlgCanonicalization"
    URI="http://www.w3.org/Signature/Drafts/xml-exc-c14n"/>
  <wsse:Algorithm Type="wsse:AlgSignature"
    URI=" http://www.w3.org/2000/09/xmldsig#rsa-sha1"/>
  <wsse:SecurityToken>
    <wsse:TokenType>wsse:X509v3</wsse:TokenType>
  </wsse:SecurityToken>
  <MessageParts 
    Dialect="http://schemas.xmlsoap.org/2002/12/wsse#soap">
    \xmlemph{S:Body some-URI:HeaderBlockElementName}
  </MessageParts>
</wsse:Integrity>
\end{xml}
\end{slide}

\begin{slide}
\Subheading{More examples}
%Policy on message age:
%\begin{xml}
%<wsse:MessageAge wsse:Usage="wsp:Required" Age="3600"/>
%\end{xml}

Policy on encodings:
\begin{xml}
<wsp:TextEncoding wsp:Usage="wsp:Required" Encoding="iso-8859-5"/>
\end{xml}

Policy on number of security headers in SOAP message:
\begin{xml}
<wsp:MessagePredicate wsp:Usage="wsp:Required">
     count(wsp:GetHeader(.)/wsse:Security) = 1
</wsp:MessagePredicate>
\end{xml}

Compound policy:
\begin{xml}
<wsp:All>
  <wsp:TextEncoding wsp:Usage="wsp:Required" Encoding="iso-8859-5"/>
  <wsp:MessagePredicate wsp:Usage="wsp:Required" .../>
</wsp:All>
\end{xml}
\small
(from \url{http://www.developer.com/design/article.php/10925_2171031_2})
\end{slide}

\begin{slide}
  \Listofslides
\end{slide}

\begin{slide}
  \Timetable
\end{slide}

\end{document}

\begin{slide}
	\Subheading{SAML: Security Assertion Markup Language}

	\begin{itemize}
	\item assertion language to allow `issuers' to make assertions about
	   \begin{itemize}
	   \item authentication: The specified subject was
	   authenticated by a particular means at a particular time.

	   \item authorization: A request to allow the specified
	   subject to access the specified resource has been granted
	   or denied.

	   \item attribute: The specified subject is associated with the
	   supplied attributes. 
	   \end{itemize}
	\item supported by a request--response protocol for communicating
		with SAML authorities

	\item assertions, requests, and responses may be signed using
	XML Signature.  
	\end{itemize}
\end{slide}

\begin{slide}
	Thus, gives a way to write architecture/application-neutral
	statements like
	\begin{itemize}
	\item \emph{user \keyword{Bob} was authenticated to \keyword{this server}
		at \keyword{date/time} by means of a password}
	\item \emph{user \keyword{Bob} was authenticated to \keyword{this server}
		at \keyword{date/time} by demonstrating ownership of
		the private key matching \keyword{public key}}
	\item \emph{user \keyword{Alice} has a credit limit of \keyword{this much}
		at \keyword{date/time}}
	\item \emph{user \keyword{Charlotte} has sysadmin privilege
		from \keyword{date/time} to \keyword{date/time}}
	\end{itemize}

	BUT the semantics (and detailed vocabulary) must be defined
	elsewhere.
\end{slide}
\begin{slide}

	Separate notions of 
	\begin{itemize}	
	\item \emph{Policy Decision Point} (PDP) and 
	\item \emph{Policy Enforcement Point} (PEP).
	\end{itemize}

	WS-SecurityPolicy has elements to carry SAML assertions---
	\xmlfrag{<wsse:SAMLAssertion>} as one of the possible security
	token types.

	Various usage scenarios.

\end{slide}

\begin{slide}
	\Subheading{XACML: eXtensible Access Control Markup Language}

	Policy Administration Point (PAP) 

	Policy Information Point (PIP)

	Policy: set of rules, rule combining algorithm

	Arriving at, and propagating the kind of authorization decisions
	that might be recorded in a SAML assertion.

\end{slide}
\begin{slide}
	\Subheading{XACML Requirements}

	\begin{itemize}
\item To provide a method for combining individual rules and policies into
a single policy set that applies to a particular decision
request.

\item To provide a method for dealing with multiple subjects acting in
different capacities. 

\item To provide a method for basing an authorization decision on
attributes of the subject and  resource. 

\item To provide a set of logical and mathematical operators on attributes
of the subject, resource and environment. 

\item To provide a method for handling a distributed set of policy
components, while abstracting the method for locating, retrieving
and authenticating the policy components. 
\item etc.
\end{itemize}
\end{slide}

\begin{slide}
	\begin{center}
	\includegraphics[width=0.7\linewidth]{xacml-process}
	\end{center}
\end{slide}

\begin{slide}
	\Example

The Standard
{\footnotesize\uri{http://www.oasis-open.org/committees/download.php/2406/oasis-xacml-1.0.pdf}}
offers a many-paged example based on fine-grained access to medical
records.

\medskip

70 lines of XML are used to describe a context 
in which a physician wishes to read the date of birth from a
particular patient record.

\medskip

The example supposes that the following rules have been created by one
or more PAPs.  \emph{Each} is expressed in 50--100 lines of XML.
\end{slide}

\begin{slide}
\begin{itemize}
\item Rule 1: A person, identified by his or her patient number, may read any record for which he
or she is the designated patient. 

\item Rule 2: A person may read any record for which he or she is the
designated parent or guardian, and for which the patient is under
16~years of age.

\item Rule 3: A physician may write to any medical element for which
he or she is the designated primary care physician, provided an email
is sent to the patient.

\item Rule 4: An administrator shall not be permitted to read or write
to medical elements of a  patient record. 
\end{itemize}

\end{slide}

\begin{slide}
	\Subheading{Discussion}

	The intention: fine-grained authorization management
	over heterogeneous domains.

	Will it ever work?  How will it work in practice?

	How are we to validate these (formal expressions of) rules?
\end{slide}


\begin{slide}
  \heading{Architecture and Security}

Role and purpose

\includegraphics[width=\linewidth]{secuirty-bigpic}

\end{slide}


\begin{slide}
  \Example[Shibboleth]

Concepts:
\begin{itemize}
  \item Identity Provider
  \item Service Provider
  \item Where are you from?
  \item Federations
\end{itemize}

\end{slide}

\begin{slide}
  \Subheading{Shibboleth Goals}

\begin{itemize}
  \item scalability
  \item single sign-on
  \item authentication near user
  \item authorization near resource
  \item user's home manages their attribuutes
  \item resource takes decision based on attributes: user may be 
    anonymous
  \end{itemize}
\end{slide}

\begin{slide}
	\begin{center}
  \includegraphics[width=0.7\linewidth]{shibboleth}
%  \includegraphics[width=0.7\linewidth]{alt-shib-figure}
	\end{center}
\end{slide}

\begin{slide}
1. User attempts to access Shibboleth-protected resource on SP site
   application server.

2, 3, 4. User is redirected to a Where Are You From (WAYF) server,
where the user indicates their home site (IdP).

5. User is redirected to the Handle Service at their IdP.

6, 7 User authenticates at their IdP, using local credentials.

\newslide

8. Handle service generates unique ID (Handle) and redirects user to
   Service Provider site's Assertion Consumer Service (ACS). ACS
   validates the supplied assertion, creates a session, and transfers
   to Attribute Requestor (AR).

9, 10. AR uses the Handle to request attributes from the IdP site's
Attribute Authority. The attribute authority responds with an
attribute assertion subject to attribute release policies; SP site
uses attributes for access control and other application-level
decisions.

{\tiny Copyright \copyright 1996-2005 Internet2}
\end{slide}

\begin{slide}
  \Subheading{Brokered trust?}

In a multi-level service assembly, there are two significant options:
\begin{itemize}
\item each service interprets the caller's security tokens, and if the
  caller is authorized, \emph{calls the subordinate services with its
  own security token(s)}.
\item each service simply passes on the caller's security tokens to
  the subordinate services; `leaf nodes' must determine if the caller
  is authorized to use the service.
\end{itemize}

\textcolor{QuestionColor}{Which is preferable?  In which
  circumstances?}

\end{slide}

