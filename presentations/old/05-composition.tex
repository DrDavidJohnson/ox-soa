\newif\iffaked \fakedfalse
\iffaked
\documentclass{sepslide-soa-faked} % jim has mucked up xdvi settings
\else
\documentclass{sepslide-soa}
\fi

\title{Composition}
\topic{5}{Composition}[Green]
\indexfile{00-index.pdf}

\begin{document}

\begin{slide}
  \Title
\end{slide}

\begin{slide}
  \Contents
\end{slide}

\begin{slide}
\Heading{Composition of services}
\begin{itemize}
\item services provide \emph{platform- and language-independent access}
  to \emph{software components}
\item but these components are \emph{isolated}: they need to be
  \emph{assembled} into \emph{service-oriented architectures}
\item ideally, they should be recursively \emph{composable} to form
  composite services in their own right
\item \emph{workflow} languages for scripting or `glue' between individual services
\item beyond mere \emph{business protocol specifications} like RosettaNet,
  which are essentially paper specifications so can't be automated and
  won't scale  
\item BPMN, WSCI, WSFL, XLANG, BPEL\ldots
\end{itemize}
\end{slide}

\begin{slide}
\Subheading{Workflow}
\begin{itemize}
\item \emph{individual} services can be represented in WSDL
\item but how should they be \emph{combined}?
\item textual documentation can explain the protocol;
  traditional programming language can implement it~---
  but then the business process is \emph{embedded in code}
\item the process has value in its own right, and should be recorded
separately as a \emph{workflow}
\item can then be stored, manipulated, and modified independently of 
individual activities
\item `don't let your business's widget-making knowledge walk out of the
door each evening in the chief widget-maker's head' (Virdell)
% Margie Virdell, "Business processes and workflow in the Web Services world", IBM developerWorks, January 2003
\item agility is the key here (c.f. Sessions on MiG-15 vs F-86)
\end{itemize}
\end{slide}

\begin{slide}
\Subheading{Removal of dependencies (Leymann and Roller)}
% F. Leymann and D. Roller, "Workflow-based applications"
% IBM Systems Journal 36(1) 1997

DBMS provides independence from data \emph{representation}; \\
workflow provides independence from control or data \emph{flow}.
\begin{center} 
\includegraphics[height=0.7\textheight]{diagrams/applicn-architecture}
\end{center}
\end{slide}

\begin{slide}
\Subheading{An architectural analogy}
\begin{center} 
\includegraphics[width=\textwidth]{diagrams/plumbing}
\end{center}
\emph{exogenous connectivity}
\end{slide}

\begin{slide}
\Subheading{Historical precedents: EAI, WfMS}
\begin{itemize}
\item \emph{enterprise application integration} (EAI)
\item resolving heterogeneity, typically via asynchronous \emph{message
    brokers} 
\item \emph{workflow management systems} (WfMS): automating interactions
\item origins in \emph{office automation}: admin processes
\item \emph{production workflows}: from information between people to
  integration of systems 
\item often associated with \emph{business process re-engineering}:
  assessment, analysis, modelling, definition, implementation
\item service composition = EAI + WfMS
\end{itemize}
\end{slide}

\begin{slide}
\Subheading{Sequencing}
\begin{itemize}
\item control-dependencies between activities: \\
  predecessor must finish before successor can start
\item data-dependencies between activities: \\
  output from predecessor is (transformed into) input to successor
\item conditional constructions: internal or external choice
\item time-out constructs
\end{itemize}
\end{slide}

\begin{slide}
\Subheading{Exception handling}
\begin{itemize}
\item consequences of exceptional conditions
\item recovery sequences
\item must return system to consistent state, wherever exception occurred
\item at least as important as the `normal' cases
\item typically expands model by factor of 4 or 5!
\end{itemize}
\end{slide}

\begin{slide}
\Subheading{Orchestration and choreography}
\begin{itemize}
\item \emph{orchestration} 
\begin{itemize}
\item describes procedure
\item instructs participants globally
\item imperative; centralized
\item typically deterministic: `must'
\end{itemize}
\item \emph{choreography} 
\begin{itemize}
\item describes protocol
\item constraints on interaction, but participants act locally
\item declarative; no `current state'
\item usually non-deterministic: `may'
\end{itemize}
\item by analogy: orchestra has conductor, ballet does not
\end{itemize}
\end{slide}

\begin{slide}
\Subheading{Compensation}
\begin{itemize}
\item business processes are usually \emph{long-running transactions}
\item want to maintain ACID properties (atomicity, consistency, isolation,
durability) 
\item \ldots but cannot afford to use locking to achieve these
\item therefore might have to compensate (as much as possible) for
cancelled transactions 
\item compensations usually assume forward transaction completed successfully
  (no `partial compensations')
\end{itemize}
\end{slide}

\begin{slide}
\Subheading{Correlation}
\begin{itemize}
\item business process typically involves some \emph{conversation} between
components 
\item therefore the same instance of a component should be involved
throughout the process (at least logically)
\item in the OO world, this is handled by \emph{object references}
\item but components should maintain no persistent state
\item so subsequent messages in a workflow should be \emph{correlated}
\item `customer reference number' or `case number'
\end{itemize}
\end{slide}

\begin{slide}
\Subheading{Polarity}
\begin{itemize}
\item WSDL operations are of four kinds:
  \begin{description}
  \item[one-way:] port receives \xmlfrag{<input>} messages only
  \item[request-response:] port receives \xmlfrag{<input>} then 
    sends~\rlap{\xmlfrag{<output>}}
  \item[notification:] port sends \xmlfrag{<output>} message only
  \item[solicit-response:] port sends \xmlfrag{<output>} then
    receives~\rlap{\xmlfrag{<input>}}
  \end{description}
\item one-way and request-response operations are \emph{incoming},
  solicit-response and notification operations \emph{outgoing}
\item workflow specification will match up ports; obviously, matched ports
  must have operations of opposing \emph{polarities}
\item WSDL does not restrict port types to aggregating operations of
  a common polarity;
  XLANG spec argues that \emph{allowing a mixture of incoming and
    outgoing operation types in a port type is highly undesirable}
\end{itemize}
\end{slide}

\begin{slide}
\Subheading{Workflow management}
\begin{itemize}
\item if workflow model is sufficiently \emph{precise and complete},
   then \emph{workflow engine} can execute process description automatically
\item workflow model dictates ordering, dataflow, compensation,\rlap{~etc}
\item well-established R\&D area (\emph{Workflow Management Coalition}
  (\url{http://www.wfmc.org/}) since early 1990s; OMG \emph{Workflow
    Management Facility Specification} 
  since 1996, version~1.2 in 2000)
\item eg \emph{megaprogramming}:  module assembly (Wiederhold, 1992)
\item now being assimilated into SOA worldview
\end{itemize}
\end{slide}

\begin{slide}
\Subheading{WS-related standards for workflow}
\begin{itemize}
\item \emph{Web Services Flow Language} (WSFL) (IBM; 2001)
\item \emph{XLANG} (Microsoft; 2001)
\item \emph{Web Services Choreography Interface} (WSCI)
(BEA, Intalio, SAP, Sun; 2002)
\item combined into \emph{Business Process Execution Language for Web
    Services} (BPEL4WS, or just BPEL) (IBM, Microsoft, BEA, SAP, Siebel;
  2003), to be submitted to OASIS
\item various ad-hoc and vendor-specific graphical notations; OMG standard \emph{Business Process Modeling Notation} (BPMN) an alternative
\end{itemize}
\end{slide}

\begin{slide}
\Heading{UML activity diagrams for recording workflow}
\begin{itemize}
\item activity diagrams illustrate \emph{time-} or \emph{data-dependencies}
  between \emph{activities}
\item based on Petri nets
\item some similarities with flowcharts,
  but are also well-suited to presenting \emph{parallel} sequences of
  activities
\item mainly used in UML to document complex operations
\item \ldots but also useful for expressing \emph{business
  workflows} 
\item approach based on Fowler, \textit{UML Distilled}
\item (there's now a draft UML profile for BPEL)
\end{itemize}
\end{slide}

\begin{slide}
\Subheading{Example activity diagram (after UML spec)}

\centerline{\includegraphics[width=\textwidth]{diagrams/coffee}}
\end{slide}

\begin{slide}
\Subheading{Activities}
\begin{itemize}
\item an atomic action, eg execution of an operation
\item with duration, and with possible interruption points
\item denoted by an oval: straight top and bottom, round ends
\end{itemize}
  \bigskip
  \centerline{\includegraphics[scale=1.2]{diagrams/activity1}} 
\end{slide}

\begin{slide}
\Subheading{Transitions}
\begin{itemize}
\item movement from one activity to another
\item normally occurs when all the actions of an activity have been
  completed
\item may occur when an event triggers the exit from a particular activity
\item indicated by an arrow between activities
\end{itemize}
\end{slide}

\begin{slide}
\Subheading{Decision points}
\begin{itemize}
\item a point where the exit
  transition from an activity can branch in alternative directions
  depending on a condition
\end{itemize}
  \bigskip
  \centerline{\includegraphics[scale=0.7]{diagrams/branch}}
\end{slide}

\begin{slide}
\Subheading{Synchronization bars}

  \begin{minipage}{0.6\textwidth} \raggedright
  \begin{itemize}
  \item split a transition into multiple
    paths, and combine multiple paths into a single transition
  \item \emph{fork} vertices have a single incoming transition
  \item \emph{join} vertices have a single outgoing transition
  \item in general, synchronization bar may have multiple incoming and
    outgoing transitions
  \item forks and joins should match up (`token balance')
  \end{itemize}
  \end{minipage}
  \qquad
  \begin{minipage}{0.3\textwidth}
  \includegraphics[scale=0.6]{diagrams/synchronization}
  \end{minipage}
\end{slide}

\begin{slide}
\Subheading{Start and end states}
\begin{itemize}
\item \emph{start state} is the entry point to a flow
\item only one start state allowed 
  \bigskip
  \centerline{\includegraphics[scale=0.7]{diagrams/startState}}
\item end states indicate the exit point from a flow
\item multiple end states are allowed
  \bigskip
  \centerline{\includegraphics[scale=0.7]{diagrams/finishState}}
\end{itemize}
\end{slide}

\begin{slide}
\Subheading{Swimlanes}
\begin{itemize}
\item partition an activity diagram into the responsibilities
  of different entities
\end{itemize}
  \bigskip
  \centerline{\includegraphics[height=0.7\textheight]{diagrams/swimlanes}}
\end{slide}

\begin{slide}
\Subheading{Dataflow}
\begin{itemize}
\item transitions between activities represent \emph{control dependencies}: one
activity must complete before another can start
\item workflows also have \emph{data dependencies}: one activity produces a
result that another requires
\item UML activity diagrams allow \emph{object flow} as well as
\emph{control flow}
\item dependent data is shown as an object icon 
(rectangle with underlined name and type)
\item \emph{dependencies} shown as dashed arrows from generating activity to
object, and from object to consuming activity(s)
\item same object may occur multiple times in an activity diagram,
  typically in different \emph{states} (shown in square brackets after
  object name)
\end{itemize}
\end{slide}

\begin{slide}
\Subheading{Example of object flow}

\centerline{\includegraphics[height=0.9\textheight]{diagrams/objectflow-assignment}}
\end{slide}

\begin{slide}
\Heading{Business Process Modelling Notation (BPMN)}
\begin{itemize}
\item graphical notation specifically for workflow
\item developed by Business Process Management Initiative, now part of OMG
\item currently v1.1; working on v2.0 (two different ones!)
\item spec also defines mapping to BPEL
\item my student Peter Wong translating to CSP
\end{itemize}
\end{slide}

\begin{slide}
\Subheading{BPMN constructs}
Very briefly:
\begin{itemize}
\item flow objects
\begin{itemize}
\item event (start, end)
\item activity
\item gateway (branching)
\end{itemize}
\item connecting objects
\begin{itemize}
\item sequence flow (control)
\item message flow (data)
% \item association
\end{itemize}
\item swimlanes
% \item artifacts
% \begin{itemize}
\item data object
% \item group
% \item annotation
% \end{itemize}
\end{itemize}
\end{slide}

\begin{slide}
\Subheading{BPMN example}
\begin{flushleft}
\includegraphics[width=\textwidth]{diagrams/airline.eps}
\end{flushleft}
\end{slide}

\begin{slide}
\Heading{WSCI}
\begin{itemize}
\item BEA Systems, Intalio, Sun, SAP
\item describes the flow of messages exchanged with a web service 
\item expressed in terms of temporal and logical dependencies between
messages
\item features sequencing rules, correlation, exception handling and
transactions 
\item allows client to understand how to interact with service, and to
anticipate its behaviour
\item `operating manual' for a web service
\item not an executable workflow language
\end{itemize}
\end{slide}

\begin{slide}
\Subheading{Questions answered by WSCI}
\begin{itemize}
\item can messages be sent and/or received in any order?
\item what rules govern the sequencing of messages?
\item what is the relationship between incoming and outgoing messages?
\item is there a `start' and an `end' to a sequence of messages?
\item can a sequence be partially undone?
\end{itemize}
\end{slide}

\begin{slide}
\Subheading{Features provided by WSCI}
\begin{itemize}
\item \emph{message choreography}: ordering rules
\item \emph{transaction boundaries and compensation}, allowing
participation in distributed transactions
\item \emph{exception handling}, reactions to exceptional circumstances
\item \emph{thread management}: capability for concurrent conversations,
and correlation of these
\item \emph{properties and selectors} that govern observable behaviour
\item \emph{connections} between consumers and producers
\item \emph{dynamic participation}: how target identity is selected
\end{itemize}
\end{slide}

\begin{slide}
\Heading{WSFL}
\begin{itemize}
\item IBM's contribution to standardization (Frank Leymann et al.)
\item two modes of use:
  \begin{description}
  \item[flow model] describes \emph{usage pattern} of collection of web
    services
  \item[global model] describes distributed \emph{interaction pattern}
    between collection of web services
  \end{description}
\item supports recursive composition: each flow model or global model can
  in turn be represented as a web service, and further composed
\end{itemize}
\end{slide}

\begin{slide}
\Subheading{Composition}
\begin{itemize}
\item one use case for WSFL
\item enterprise implements a business process by composing existing web
  services 
\item existing base services already provided by other vendors (or by this one)
\item coordination of these services described in WSFL
\item eg providing a \textsc{Fa\c cade} for a complex collection of services
\item advert in service repository would attract offers from consumers 
\end{itemize}
\end{slide}

\begin{slide}
\Subheading{Parametrization}
\begin{itemize}
\item a second use case for WSFL, a variation on the first
\item abstract away from particular implementations of base services;
  leave \emph{plug links} for later binding
\item still use WSFL to describe the coordination
\item describes \emph{exports} (services provided) and \emph{imports}
  (services required)
\item appropriate for describing service \emph{brokers};
  eg comparative shopping service acts as a broker between shoppers and
  shops 
\item advert in service repository would attract two kinds
  of offers: from producers and consumers
\end{itemize}
\end{slide}

\begin{slide}
\Heading{XLANG}
\begin{itemize}
\item Microsoft's contribution, part of BizTalk (EAI for .Net)
\item sequential and parallel flow constructs
\item internal and external choice
\item long running transactions with compensations
\item custom correlation of messages
\item exception handling
\item abstraction from detail: \emph{opaque values}
\end{itemize}
\end{slide}

\begin{slide}
\Heading{WS-BPEL}
\begin{itemize}
\item formerly \emph{Business Process Execution Language for Web Services}
\item a combination of IBM's WSFL (graph-oriented control flow) and
  Microsoft's XLANG (structured control flow)
\item input also from BEA, Siebel, SAP, Sun
\item envisage two usage patterns: for describing \emph{abstract processes}
  (describing interaction patterns but not data transformations) and
  \emph{executable processes} 
\item common core, with extensions for each usage pattern
\end{itemize}
\end{slide}

\begin{slide}
\Subheading{Executable processes vs business protocols}
\begin{itemize}
\item \emph{executable business process} captures participants' exact
behaviour
\item \emph{business protocol} is an abstraction of this: describes message
exchanges but not specific data
\item business protocol involves \emph{transparent data} (relevant to
public aspects) and \emph{opaque data} (relevant only to private back-end
systems)
\item analogous to network packet's \emph{control fields} and
\emph{payload}
\item from business protocol point of view, opaque data is
nondeterministically chosen
\item a contribution from XLANG (which provided only business protocols,
  but anticipated extensions to executable processes)
\end{itemize}
\end{slide}

\begin{slide}
\Subheading{Simple example}
\begin{itemize}
\item purchase order system
\item system receives purchase order from customer
\item price is calculated, shipping arranged, production scheduled
  (in parallel)
\item on completion, invoice is produced
\item choice of shipper feeds into price calculation
\item shipping date feeds into production schedule
\medskip
\item see appendix for WSDL and BPEL
\end{itemize}
\end{slide}

\begin{slide}
\Subheading{Abstract activity diagram for purchase orders}
\begin{center}
\includegraphics[height=0.9\textheight]{diagrams/order-activity}
\end{center}
\end{slide}

\begin{slide}
\Subheading{Messages and port types}
\begin{center}
\includegraphics[height=0.9\textheight]{diagrams/order-class}
\end{center}
\end{slide}

\begin{slide}
\Subheading{Partner links}
\begin{center}
\includegraphics[height=0.9\textheight]{diagrams/order-component}
\end{center}
\end{slide}

\begin{slide}
\Subheading{Notes on partner links}
\begin{itemize}
\item interactions between business process and external parties
\item \emph{partner link type} defines one (for unidirectional links) or
  two (for bidirectional) \emph{roles} of corresponding \emph{port~types}
\item (here, no requirements are placed on the customer using the
  \textit{PurchaseOrder} component, and no requirements are placed on this
  for using the \textit{ProductionScheduler} component; for an asynchronous
  service, customer would have to provide a callback interface)
\item \emph{partner links} instantiate partner link types, specifying
  \textit{myRole} (played by this process) and/or \textit{partnerRole}
  (played by external party)
\item bindings of actual partners to external roles are omitted
\item would expect opposites to match up, but WS-BPEL supports only incoming
\end{itemize}
\end{slide}

\begin{slide}
\Subheading{Coordination}
\begin{center}
\includegraphics[height=0.9\textheight]{diagrams/order-flow}
\end{center}
\end{slide}

\begin{slide}
\Subheading{Notes on coordination}
\begin{itemize}
\item dataflow links by global \emph{variables}
\item \emph{fault handlers} for exceptions (omitted in activity diagram for
  simplicity) 
\item \xmlfrag{<sequence>} element for sequential ordering
\item \xmlfrag{<flow>} element for concurrent execution
\item \xmlfrag{<link>} elements for extra dependencies within \xmlfrag{<flow>}
\item simple \xmlfrag{<assign>}ments shown; more complex ones available
\item control constructs (\xmlfrag{<switch>}, \xmlfrag{<while>}) also available
\item \xmlfrag{<receive>} for messages consumed by process, and
  \xmlfrag{<reply>} for responses
\item \xmlfrag{<invoke>} for messages sent to external parties
\end{itemize}
\end{slide}

\begin{slide}
\Heading{Reflection}
\begin{itemize}
\item BPEL represented in XML is very ugly; not worth exploring here
\item tools should do all the relevant translating for you: UML to
BPEL translators exist, for example
\item we can use this approach to describe abstract and concrete
coordination processes;  the concrete coordination process has to run
somewhere
\item design activity is the crucial piece~--- UML profiles
can help
\end{itemize}
\end{slide}

\begin{slide}
  \Listofslides
\end{slide}

\begin{slide}
  \Timetable
\end{slide}

\end{document}
